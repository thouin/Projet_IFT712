\chapter{Analyse des résultats}
\par Dans cette partie nous allons analyser les résultats de chacun des algorithmes.
\section{Résultats}
\subsection{Régression logistique}
\begin{figure}[H]
    \centering
    \includegraphics[width=17cm, height=10cm, keepaspectratio]{logistique_plot.png}
    \caption{Figure diagrames "accuracy" et "loss" pour la régression logistique }
    \label{Figure diagrames "accuracy" et "loss" pour la régression logistique  }
\end{figure}
\par Pour "lr" entre 0.0001 et 0.001, et pour "l2reg" entre 0.1 et 10, les meilleurs valeurs pour les hyper-paramètres "lr" et "l2reg" sont ceux affichés dans la figure ci-dessous
\begin{figure}[H]
    \centering
    \includegraphics{logistique_param.PNG}
    \caption{Figure parametres de régression logistique}
    \label{Figure parametres de régression logistique }
\end{figure}
\subsection{SVM}
\par Les erreurs et les précisions d'entrainement et de test:
\begin{figure}[H]
    \centering
    \includegraphics{svm.png}
    \caption{Figure SVM-error }
    \label{Figure fichier svm-error }
\end{figure}
\subsection{Réseaux de neuronnes}
\begin{figure}[H]
    \centering
    \includegraphics[width=17cm, height=10cm, keepaspectratio]{rn_plot.png}
    \caption{Figure diagrames "accuracy" et "loss" pour réseau de neuronne }
    \label{Figure diagrames "accuracy" et "loss" pour réseau de neuronne }
\end{figure}   
\par Pour "lr" entre 0.01 et 1, et pour "l2reg" entre 0.1 et 10, et pour "mu" entre 0 et 1, les meilleurs valeurs pour les hyper-paramètres "lr", "l2reg" et "mu" sont ceux affichés dans les deux figures ci-dessous
\begin{figure}[H]
    \centering
    \includegraphics{rn_param_logistic.PNG}
    \caption{Figure paramètres pour une fonction d'activation logistique}
    \label{paramètres pour une fonction d'activation logistique }
\end{figure}
\begin{figure}[H]
    \centering
    \includegraphics{rm_param_relu.PNG}
    \caption{Figure paramètres pour une fonction d'activation RELU}
    \label{paramètres pour une fonction d'activation RELU }
\end{figure}
\subsection{Bagging}
\begin{figure}[H]
    \centering
    \includegraphics{bagging_param.PNG}
    \caption{Figure bagging-param }
    \label{Figure fichier bagging-param }
\end{figure}
\par Les précisions d'entrainement et de test:
\begin{figure}[H]
    \centering
    \includegraphics{bagging.PNG}
    \caption{Figure bagging-error }
    \label{Figure fichier bagging-error }
\end{figure}
\subsection{AdaBoost} 
\par Pour "lr" entre 0.01 et 1, et pour "n-estimators" entre 1 et 101, les meilleurs valeurs pour les hyper-paramètres "lr" et "n-estimator" sont ceux affichés dans la figure ci-dessous
\begin{figure}[H]
    \centering
    \includegraphics{adaboost_param.PNG}
    \caption{Figure adaboost-param }
    \label{Figure fichier adaboost-param }
\end{figure}
\par Les précisions d'entrainement et de test:
\begin{figure}[H]
    \centering
    \includegraphics{adaboost.PNG}
    \caption{Figure adaboost-error }
    \label{Figure fichier adaboost-error }
\end{figure}
