\chapter{Présentation du projet}
\section{Présentation}
\par Ce projet de session fait partie des travaux du cours IFT712, il a pour objectif de tester quelques méthodes de classification sur une base de données
Kaggle (www.kaggle.com) avec la bibliothèque Sklearn .
\section{Choix de la base de données}
\par Nous avons choisi comme base de données: "Wine Dataset".
\par A propos de la base de données, les données sont le résultat d'une analyse chimique de vins cultivés dans une des régions d'Italie mais issus de trois cultivars différents.
\par Parmi ses caracteristiques:
\begin{itemize}[label=\textbullet]
    \item 178 instances
    \item 13 variables
    \item les valeurs des attributs sont des Integer et des Float
    \item Pas de valeurs manquantes
    \item Généralement utilisé pour les tâches de classification
\end{itemize}
\section{Choix du design}
\par Le projet contient 8 classes en total.
\begin{itemize}[label=\textbullet]
    \item La classe 'Classification\_main.py'
    \item La classe 'Classification\_neural\_net.py': Décrit l'algorithme de classification par réseau de neurones.
    \item La classe 'Classification\_logistique.py': S'agit de la classification par régréssion logistique.
    \item La classe 'Classification\_bagging.py': Classification par l'algorithme Bagging.
    \item La classe 'Classification\_adaboost.py': Classification par l'algorithme AdaBoost.
    \item La classe 'Classification\_svm.py': Il s'agit du code correspondant au SVM.
    \item La classe 'Classification\_hyperparameter.py': Il s'agit des techniques de la recherche des hypers paramètres pour chacun des algorithmes de classification.
    \item La classe 'Classification\_io.py': Cette classe crée des données d"entrainement et de test à partir du "Wine Dataset" et a une fonction permettant l'affichage graphique.
\end{itemize}
\section{Gestion du projet}
\par Pour la réalisation du projet, nous avons utilisé le gestionnaire de version de code "git" via la plateforme "gitHub"
